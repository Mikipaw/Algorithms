\documentclass[12pt]{extreport}
\usepackage[T2A]{fontenc}
\usepackage[utf8]{inputenc}        % Кодировка входного документа;
                                    % при необходимости, вместо cp1251
                                    % можно указать cp866 (Alt-кодировка
                                    % DOS) или koi8-r.

\usepackage[english,russian]{babel} % Включение русификации, русских и
                                    % английских стилей и переносов
%%\usepackage{a4}
%%\usepackage{moreverb}
\usepackage{amsmath,amsfonts,amsthm,amssymb,amsbsy,amstext,amscd,amsxtra,multicol}
\usepackage{indentfirst}
\usepackage{verbatim}
\usepackage{tikz} %Рисование автоматов
\usetikzlibrary{automata,positioning}
\usepackage{multicol} %Несколько колонок
\usepackage{graphicx}
\usepackage[colorlinks,urlcolor=blue]{hyperref}
\usepackage[stable]{footmisc}

%% \voffset-5mm
%% \def\baselinestretch{1.44}
\renewcommand{\theequation}{\arabic{equation}}
\def\hm#1{#1\nobreak\discretionary{}{\hbox{$#1$}}{}}
\newtheorem{Lemma}{Лемма}
\theoremstyle{definiton}
\newtheorem{Remark}{Замечание}
%%\newtheorem{Def}{Определение}
\newtheorem{Claim}{Утверждение}
\newtheorem{Cor}{Следствие}
\newtheorem{Theorem}{Теорема}
\theoremstyle{definition}
\newtheorem{Example}{Пример}
\newtheorem*{known}{Теорема}
\def\proofname{Доказательство}
\theoremstyle{definition}
\newtheorem{Def}{Определение}

%% \newenvironment{Example} % имя окружения
%% {\par\noindent{\bf Пример.}} % команды для \begin
%% {\hfill$\scriptstyle\qed$} % команды для \end






%\date{22 июня 2011 г.}
\let\leq\leqslant
\let\geq\geqslant
\def\MT{\mathrm{MT}}
%Обозначения ``ажуром''
\def\BB{\mathbb B}
\def\CC{\mathbb C}
\def\RR{\mathbb R}
\def\SS{\mathbb S}
\def\ZZ{\mathbb Z}
\def\NN{\mathbb N}
\def\FF{\mathbb F}
%греческие буквы
\let\epsilon\varepsilon
\let\es\varnothing
\let\eps\varepsilon
\let\al\alpha
\let\sg\sigma
\let\ga\gamma
\let\ph\varphi
\let\om\omega
\let\ld\lambda
\let\Ld\Lambda
\let\vk\varkappa
\let\Om\Omega
\def\abstractname{}

\def\R{{\cal R}}
\def\A{{\cal A}}
\def\B{{\cal B}}
\def\C{{\cal C}}
\def\D{{\cal D}}

%классы сложности
\def\REG{{\mathsf{REG}}}
\def\CFL{{\mathsf{CFL}}}


%%%%%%%%%%%%%%%%%%%%%%%%%%%%%%% Problems macros  %%%%%%%%%%%%%%%%%%%%%%%%%%%%%%%


%%%%%%%%%%%%%%%%%%%%%%%% Enumerations %%%%%%%%%%%%%%%%%%%%%%%%

\newcommand{\Rnum}[1]{\expandafter{\romannumeral #1\relax}}
\newcommand{\RNum}[1]{\uppercase\expandafter{\romannumeral #1\relax}}

%%%%%%%%%%%%%%%%%%%%% EOF Enumerations %%%%%%%%%%%%%%%%%%%%%

\usepackage{xparse}
\usepackage{ifthen}
\usepackage{bm} %%% bf in math mode
\usepackage{color}
%\usepackage[usenames,dvipsnames]{xcolor}

\definecolor{Gray555}{HTML}{555555}
\definecolor{Gray444}{HTML}{444444}
\definecolor{Gray333}{HTML}{333333}


\newcounter{problem}
\newcounter{uproblem}
\newcounter{subproblem}
\newcounter{prvar}

\def\beforPRskip{
	\bigskip
	%\vspace*{2ex}
}

\def\PRSUBskip{
	\medskip
}


\def\pr{\beforPRskip\noindent\stepcounter{problem}{\bf \theproblem .\;}\setcounter{subproblem}{0}}
\def\pru{\beforPRskip\noindent\stepcounter{problem}{\bf $\mathbf{\theproblem}^\circ$\!\!.\;}\setcounter{subproblem}{0}}
\def\prstar{\beforPRskip\noindent\stepcounter{problem}{\bf $\mathbf{\theproblem}^*$\negthickspace.}\setcounter{subproblem}{0}\;}
\def\prpfrom[#1]{\beforPRskip\noindent\stepcounter{problem}{\bf Задача \theproblem~(№#1 из задания).  }\setcounter{subproblem}{0} }
\def\prp{\beforPRskip\noindent\stepcounter{problem}{\bf Задача \theproblem .  }\setcounter{subproblem}{0} }

\def\prpvar{\beforPRskip\noindent\stepcounter{problem}\setcounter{prvar}{1}{\bf Задача \theproblem \;$\langle${\rm\Rnum{\theprvar}}$\rangle$.}\setcounter{subproblem}{0}\;}
\def\prpv{\beforPRskip\noindent\stepcounter{prvar}{\bf Задача \theproblem \,$\bm\langle$\bracketspace{{\rm\Rnum{\theprvar}}}$\bm\rangle$.  }\setcounter{subproblem}{0} }
\def\prv{\beforPRskip\noindent\stepcounter{prvar}{\bf \theproblem\,$\bm\langle$\bracketspace{{\rm\Rnum{\theprvar}}}$\bm\rangle$}.\setcounter{subproblem}{0} }

\def\prpstar{\beforPRskip\noindent\stepcounter{problem}{\bf Задача $\bf\theproblem^*$\negthickspace.  }\setcounter{subproblem}{0} }
\def\prdag{\beforPRskip\noindent\stepcounter{problem}{\bf Задача $\theproblem^{^\dagger}$\negthickspace\,.  }\setcounter{subproblem}{0} }
\def\upr{\beforPRskip\noindent\stepcounter{uproblem}{\bf Упражнение \theuproblem .  }\setcounter{subproblem}{0} }
%\def\prp{\vspace{5pt}\stepcounter{problem}{\bf Задача \theproblem .  } }
%\def\prs{\vspace{5pt}\stepcounter{problem}{\bf \theproblem .*   }
\def\prsub{\PRSUBskip\noindent\stepcounter{subproblem}{\sf \thesubproblem .} }
\def\prsubr{\PRSUBskip\noindent\stepcounter{subproblem}{\bf \asbuk{subproblem})}\;}
\def\prsubstar{\PRSUBskip\noindent\stepcounter{subproblem}{\rm $\thesubproblem^*$\negthickspace.  } }
\def\prsubrstar{\PRSUBskip\noindent\stepcounter{subproblem}{$\text{\bf \asbuk{subproblem}}^*\mathbf{)}$}\;}

\newcommand{\bracketspace}[1]{\phantom{(}\!\!{#1}\!\!\phantom{)}}

\DeclareDocumentCommand{\Prpvar}{ O{null} O{} }{
	\beforPRskip\noindent\stepcounter{problem}\setcounter{prvar}{1}{\bf Задача \theproblem
% 	\ifthenelse{\equal{#1}{null}}{  }{ {\sf $\bm\langle$\bracketspace{#1}$\bm\rangle$}}
%	~\!\!(\bracketspace{{\rm\Rnum{\theprvar}}}).  }\setcounter{subproblem}{0}
%	\;(\bracketspace{{\rm\Rnum{\theprvar}}})}\setcounter{subproblem}{0}
%
	\,{\sf $\bm\langle$\bracketspace{{\rm\Rnum{\theprvar}}}$\bm\rangle$}
	~\!\!\! \ifthenelse{\equal{#1}{null}}{\!}{{\sf(\bracketspace{#1})}}}.

}
%\DeclareDocumentCommand{\Prpvar}{ O{level} O{meta} m }{\prpvar}


\DeclareDocumentCommand{\Prp}{ O{null} O{null} }{\setcounter{subproblem}{0}
	\beforPRskip\noindent\stepcounter{problem}\setcounter{prvar}{0}{\bf Задача \theproblem
	~\!\!\! \ifthenelse{\equal{#1}{null}}{\!}{{\sf(\bracketspace{#1})}}
	 \ifthenelse{\equal{#2}{null}}{\!\!}{{\sf [\color{Gray444}\,\bracketspace{{\fontfamily{afd}\selectfont#2}}\,]}}}.}

\DeclareDocumentCommand{\Pr}{ O{null} O{null} }{\setcounter{subproblem}{0}
	\beforPRskip\noindent\stepcounter{problem}\setcounter{prvar}{0}{\bf\theproblem
	~\!\!\! \ifthenelse{\equal{#1}{null}}{\!\!}{{\sf(\bracketspace{#1})}}
	 \ifthenelse{\equal{#2}{null}}{\!\!}{{\sf [\color{Gray444}\,\bracketspace{{\fontfamily{afd}\selectfont#2}}\,]}}}.}

%\DeclareDocumentCommand{\Prp}{ O{level} O{meta} }

\DeclareDocumentCommand{\Prps}{ O{null} O{null} }{\setcounter{subproblem}{0}
	\beforPRskip\noindent\stepcounter{problem}\setcounter{prvar}{0}{\bf Задача $\bm\theproblem^* $
	~\!\!\! \ifthenelse{\equal{#1}{null}}{\!}{{\sf(\bracketspace{#1})}}
	 \ifthenelse{\equal{#2}{null}}{\!\!}{{\sf [\color{Gray444}\,\bracketspace{{\fontfamily{afd}\selectfont#2}}\,]}}}.
}

\DeclareDocumentCommand{\Prpd}{ O{null} O{null} }{\setcounter{subproblem}{0}
	\beforPRskip\noindent\stepcounter{problem}\setcounter{prvar}{0}{\bf Задача $\bm\theproblem^\dagger$
	~\!\!\! \ifthenelse{\equal{#1}{null}}{\!}{{\sf(\bracketspace{#1})}}
	 \ifthenelse{\equal{#2}{null}}{\!\!}{{\sf [\color{Gray444}\,\bracketspace{{\fontfamily{afd}\selectfont#2}}\,]}}}.
}


\def\prend{
	\bigskip
%	\bigskip
}




%%%%%%%%%%%%%%%%%%%%%%%%%%%%%%% EOF Problems macros  %%%%%%%%%%%%%%%%%%%%%%%%%%%%%%%



%\usepackage{erewhon}
%\usepackage{heuristica}
%\usepackage{gentium}

\usepackage[portrait, top=3cm, bottom=1.5cm, left=3cm, right=2cm]{geometry}

\usepackage{fancyhdr}
\pagestyle{fancy}
\renewcommand{\headrulewidth}{0pt}
\lhead{\fontfamily{fca}\selectfont {Основные алгоритмы 2022} }
%\lhead{ \bf  {Основные алгоритмы. } Павлов М.А. }
%\chead{\fontfamily{fca}\selectfont {Вариант 1}}
\rhead{\fontfamily{fca}\selectfont Павлов М.А. Домашнее задание 6}
%\rhead{\small 01.09.2016}
\cfoot{}

\usepackage{titlesec}
\titleformat{\section}[block]{\Large\bfseries\filcenter {\setcounter{problem}{0}}  }{}{1em}{}


%%%%%%%%%%%%%%%%%%%%%%%%%%%%%%%%%%%%%%%%%%%%%%%%%%%% Обозначения и операции %%%%%%%%%%%%%%%%%%%%%%%%%%%%%%%%%%%%%%%%%%%%%%%%%%%% 
                                                                    
\newcommand{\divisible}{\mathop{\raisebox{-2pt}{\vdots}}}           
\let\Om\Omega


%%%%%%%%%%%%%%%%%%%%%%%%%%%%%%%%%%%%%%%% Shen Macroses %%%%%%%%%%%%%%%%%%%%%%%%%%%%%%%%%%%%%%%%
\newcommand{\w}[1]{{\hbox{\texttt{#1}}}}

%\newcommand{\Prob}{\mathop{\mathrm{P}}}
\newcommand{\Ex}{\mathop{\mathrm{E}}}

\begin{document}
	


\pr  Игрок играет в казино в следующую игру. Делает ставку $c$, говорит крупье число от $1$ до $6$, после чего бросает три кубика. Если его число не выпало, то игрок ничего не получает, т.\,е. проигрывает 100 рублей; считаем, что в этом случае его выигрыш равен $-100$. Если же число выпало, то игрок получает свою ставку обратно и получает выигрыш~"--- за каждое выпадание числа, казино платит игроку ставку, которую он поставил. Так, если игрок поставил сто рублей и его число выпало два раза, то игрок получит выигрыш $200$ рублей, а если не выпало ни разу, то его выигрыш равен $-100$ рублей. Найдите математическое ожидание выигрыша игрока, при ставке $100$ рублей.

По формуле мат. ожадания $E = 3 \frac{1}{6} \cdot \frac{5}{6} \cdot \frac{5}{6} \cdot 100 + 3 \cdot 200 \frac{1}{36} \cdot \frac{5}{6} + 300 \frac{1}{216} - 100 \frac{125}{216} = - \frac{1700}{216} = - \frac{425}{54}$

  	\pr В лотерее на выигрыши уходит $40\%$ от стоимости проданных
	билетов. Каждый билет стоит 100 рублей. Докажите, что вероятность
	выиграть 5000 рублей (или больше) меньше $1\%$. 

	\begin{proof}
		Докажем от противного:

		Пусть $1\%$ или больше. В таком случае математическое ожидание выигрыша равно $E_0 \cdot 0.99 + 0.01 \cdot 5000 \ge 50.$
		
		Но в таком случае получается, что на каждый билет в среднем ожидается как минимум 50 рублей выигрыша, т.е. $50\%$ стоимости билета, откуда становится ясно,
		что на все выигрыши уходит более 50$\%$ стоимости проданных билетов, что противоречит условию. 
		
		Значит, предположение неверно и вероятность выиграть $\geq 5000$ рублей менее процента.

	\end{proof}

	\noindent\textbf{Замечание.} Подробности о проведении лотерии неизвестны. Приведённой информации в условии достаточно для решения задачи.

	\pr Выбирается случайное  слово длины 20 в алфавите $\{a,b\}$ (все
	слова равновозможны). Найдите математическое ожидание числа подслов
	$ab$ в этом слове.

	Решим задачу через рекурсию, найдя количество подслов ab. 

	Заметим, что при выделении подслова ab, мы будем решать аналогичную задачу, но уже для слова длины $n-2$. Это слово может стоять на $n-1$-ой позиции.

	Таким образом, число вхождений подслова ab равно $(n-1) \cdot 2^{n-2}$. Общее же количество слов $2^n$. 

	Тогда мат ожидание равно $\frac{(n-1) \cdot 2^{n-2}}{2^n} = \frac{n-1}{4}.$

	\pr \emph{Инверсией} в перестановке $a_1a_2\dots a_n$ называется такая пара
	индексов $i<j$, что $a_i>a_j$. Пусть $\pi$~--- случайная перестановка
	(все перестановки равновозможны). Найдите математическое ожидание
	$\Ex[I(\pi)]$ количества инверсий $I(\pi)$.
	
	Поскольку из курса ОВАиТК мы знаем, что для каждой перестановки существует обратная, то, очевидно, вероятность встретить инверсию равна $\frac{1}{2}$.

	В таком случае мат ожидание количества инверсий равно $\frac{1}{2} \frac{n \cdot (n-1)}{2} = \frac{n \cdot (n-1)}{4}$.

	\pr Вероятностное пространство~"--- перестановки $(x_1,\ldots,x_n)$ элементов от $1$ до $n$. Найдите математическое ожидание чисел, не поменявших своё место. Формально, случайная величина~"--- количество элементов множества $\{i \mid x_i = i\}$.
	
	Можно заметить, что эта задача изоморфна задаче о рассеянной секретарше, которая разбиралась в курсе АЛКТГ. То есть здесь вероятность, что число не поменяло свое место это вероятность, что письмо дойдет до адресата в задаче о секретарше, т.е. $\frac{1}{n}$.

	В таком случае математическое ожидание будет равно $\sum_1 ^n \frac{1}{n} = 1$.

	\pr Пусть  $X$~--- неотрицательная случайная величина. Известно, что
	$\Ex[2^X]=5$. Докажите, что $$\Prob[X\geq 6]< 1/10.$$

		\begin{proof}
			Вновь попробуем доказать от противного:

			Пусть вероятность числа $\geq 6$ больше или равна $\frac{1}{10}$.

			В таком случае математическое ожидание будет равно $E_0 + (\frac{1}{10} + \alpha) 2^{6 + \beta} \geq E_0 + \frac{1}{10} 2^6 = E_0 + 6.4 > 5$, что противоречит условию. Значит, предположение неверно и вероятность, что $X \geq 6 < \frac{1}{10}$.
		\end{proof}
	
	\pr В неориентированном графе без петель и кратных ребер графе $n$ вершин и $nd/2$ рёбер (то есть средняя степень вершины
		равна $d$), $d\geq1$. Докажите, что в графе есть независимое множество
		размера не меньше $n/2d$.
		
		И вновь воспользуемся нашими знаниями из курса АЛКТГ.

		Мы доказывали следующую теорему: $\chi (G) \geq \frac{n}{\alpha(G)}$, где $\chi$ -- это хроматическое число, а $\alpha(G)$ -- число независимости.

		Из этой теоремы $\alpha (G) \geq \frac{n}{\chi(G)}$.

		Очевидно, что при средней степени вершины $d$ максимальное значение хроматического числа при наличии клики $K_{2d}$, т.е. максимально возможное хроматическое число для графа с таким количество вершин равно $2d$. 
		
		Значит, $\alpha (G) \geq \frac{n}{2d}$, т.е. в графе найдется независимое множество не меньше $\frac{n}{2d}$.

	\textsl{Подсказка.} В решении этой задачи поможет случайное множество
	$V_p$, в которое каждая вершина входит с вероятностью $p$ независимо
	от других вершин. (При подходящем значении параметра $p$.)
\end{document}
  