\documentclass[12pt]{extreport}
\usepackage[T2A]{fontenc}
\usepackage[utf8]{inputenc}        % Кодировка входного документа;
                                    % при необходимости, вместо cp1251
                                    % можно указать cp866 (Alt-кодировка
                                    % DOS) или koi8-r.

\usepackage[english,russian]{babel} % Включение русификации, русских и
                                    % английских стилей и переносов
%%\usepackage{a4}
%%\usepackage{moreverb}
\usepackage{amsmath,amsfonts,amsthm,amssymb,amsbsy,amstext,amscd,amsxtra,multicol}
\usepackage{indentfirst}
\usepackage{verbatim}
\usepackage{tikz} %Рисование автоматов
\usetikzlibrary{automata,positioning}
\usepackage{multicol} %Несколько колонок
\usepackage{graphicx}
\usepackage[colorlinks,urlcolor=blue]{hyperref}
\usepackage[stable]{footmisc}

%% \voffset-5mm
%% \def\baselinestretch{1.44}
\renewcommand{\theequation}{\arabic{equation}}
\def\hm#1{#1\nobreak\discretionary{}{\hbox{$#1$}}{}}
\newtheorem{Lemma}{Лемма}
\theoremstyle{definiton}
\newtheorem{Remark}{Замечание}
%%\newtheorem{Def}{Определение}
\newtheorem{Claim}{Утверждение}
\newtheorem{Cor}{Следствие}
\newtheorem{Theorem}{Теорема}
\theoremstyle{definition}
\newtheorem{Example}{Пример}
\newtheorem*{known}{Теорема}
\def\proofname{Доказательство}
\theoremstyle{definition}
\newtheorem{Def}{Определение}

%% \newenvironment{Example} % имя окружения
%% {\par\noindent{\bf Пример.}} % команды для \begin
%% {\hfill$\scriptstyle\qed$} % команды для \end






%\date{22 июня 2011 г.}
\let\leq\leqslant
\let\geq\geqslant
\def\MT{\mathrm{MT}}
%Обозначения ``ажуром''
\def\BB{\mathbb B}
\def\CC{\mathbb C}
\def\RR{\mathbb R}
\def\SS{\mathbb S}
\def\ZZ{\mathbb Z}
\def\NN{\mathbb N}
\def\FF{\mathbb F}
%греческие буквы
\let\epsilon\varepsilon
\let\es\varnothing
\let\eps\varepsilon
\let\al\alpha
\let\sg\sigma
\let\ga\gamma
\let\ph\varphi
\let\om\omega
\let\ld\lambda
\let\Ld\Lambda
\let\vk\varkappa
\let\Om\Omega
\def\abstractname{}

\def\R{{\cal R}}
\def\A{{\cal A}}
\def\B{{\cal B}}
\def\C{{\cal C}}
\def\D{{\cal D}}

%классы сложности
\def\REG{{\mathsf{REG}}}
\def\CFL{{\mathsf{CFL}}}


%%%%%%%%%%%%%%%%%%%%%%%%%%%%%%% Problems macros  %%%%%%%%%%%%%%%%%%%%%%%%%%%%%%%


%%%%%%%%%%%%%%%%%%%%%%%% Enumerations %%%%%%%%%%%%%%%%%%%%%%%%

\newcommand{\Rnum}[1]{\expandafter{\romannumeral #1\relax}}
\newcommand{\RNum}[1]{\uppercase\expandafter{\romannumeral #1\relax}}

%%%%%%%%%%%%%%%%%%%%% EOF Enumerations %%%%%%%%%%%%%%%%%%%%%

\usepackage{xparse}
\usepackage{ifthen}
\usepackage{bm} %%% bf in math mode
\usepackage{color}
%\usepackage[usenames,dvipsnames]{xcolor}

\definecolor{Gray555}{HTML}{555555}
\definecolor{Gray444}{HTML}{444444}
\definecolor{Gray333}{HTML}{333333}


\newcounter{problem}
\newcounter{uproblem}
\newcounter{subproblem}
\newcounter{prvar}

\def\beforPRskip{
	\bigskip
	%\vspace*{2ex}
}

\def\PRSUBskip{
	\medskip
}


\def\pr{\beforPRskip\noindent\stepcounter{problem}{\bf \theproblem .\;}\setcounter{subproblem}{0}}
\def\pru{\beforPRskip\noindent\stepcounter{problem}{\bf $\mathbf{\theproblem}^\circ$\!\!.\;}\setcounter{subproblem}{0}}
\def\prstar{\beforPRskip\noindent\stepcounter{problem}{\bf $\mathbf{\theproblem}^*$\negthickspace.}\setcounter{subproblem}{0}\;}
\def\prpfrom[#1]{\beforPRskip\noindent\stepcounter{problem}{\bf Задача \theproblem~(№#1 из задания).  }\setcounter{subproblem}{0} }
\def\prp{\beforPRskip\noindent\stepcounter{problem}{\bf Задача \theproblem .  }\setcounter{subproblem}{0} }

\def\prpvar{\beforPRskip\noindent\stepcounter{problem}\setcounter{prvar}{1}{\bf Задача \theproblem \;$\langle${\rm\Rnum{\theprvar}}$\rangle$.}\setcounter{subproblem}{0}\;}
\def\prpv{\beforPRskip\noindent\stepcounter{prvar}{\bf Задача \theproblem \,$\bm\langle$\bracketspace{{\rm\Rnum{\theprvar}}}$\bm\rangle$.  }\setcounter{subproblem}{0} }
\def\prv{\beforPRskip\noindent\stepcounter{prvar}{\bf \theproblem\,$\bm\langle$\bracketspace{{\rm\Rnum{\theprvar}}}$\bm\rangle$}.\setcounter{subproblem}{0} }

\def\prpstar{\beforPRskip\noindent\stepcounter{problem}{\bf Задача $\bf\theproblem^*$\negthickspace.  }\setcounter{subproblem}{0} }
\def\prdag{\beforPRskip\noindent\stepcounter{problem}{\bf Задача $\theproblem^{^\dagger}$\negthickspace\,.  }\setcounter{subproblem}{0} }
\def\upr{\beforPRskip\noindent\stepcounter{uproblem}{\bf Упражнение \theuproblem .  }\setcounter{subproblem}{0} }
%\def\prp{\vspace{5pt}\stepcounter{problem}{\bf Задача \theproblem .  } }
%\def\prs{\vspace{5pt}\stepcounter{problem}{\bf \theproblem .*   }
\def\prsub{\PRSUBskip\noindent\stepcounter{subproblem}{\sf \thesubproblem .} }
\def\prsubr{\PRSUBskip\noindent\stepcounter{subproblem}{\bf \asbuk{subproblem})}\;}
\def\prsubstar{\PRSUBskip\noindent\stepcounter{subproblem}{\rm $\thesubproblem^*$\negthickspace.  } }
\def\prsubrstar{\PRSUBskip\noindent\stepcounter{subproblem}{$\text{\bf \asbuk{subproblem}}^*\mathbf{)}$}\;}

\newcommand{\bracketspace}[1]{\phantom{(}\!\!{#1}\!\!\phantom{)}}

\DeclareDocumentCommand{\Prpvar}{ O{null} O{} }{
	\beforPRskip\noindent\stepcounter{problem}\setcounter{prvar}{1}{\bf Задача \theproblem
% 	\ifthenelse{\equal{#1}{null}}{  }{ {\sf $\bm\langle$\bracketspace{#1}$\bm\rangle$}}
%	~\!\!(\bracketspace{{\rm\Rnum{\theprvar}}}).  }\setcounter{subproblem}{0}
%	\;(\bracketspace{{\rm\Rnum{\theprvar}}})}\setcounter{subproblem}{0}
%
	\,{\sf $\bm\langle$\bracketspace{{\rm\Rnum{\theprvar}}}$\bm\rangle$}
	~\!\!\! \ifthenelse{\equal{#1}{null}}{\!}{{\sf(\bracketspace{#1})}}}.

}
%\DeclareDocumentCommand{\Prpvar}{ O{level} O{meta} m }{\prpvar}


\DeclareDocumentCommand{\Prp}{ O{null} O{null} }{\setcounter{subproblem}{0}
	\beforPRskip\noindent\stepcounter{problem}\setcounter{prvar}{0}{\bf Задача \theproblem
	~\!\!\! \ifthenelse{\equal{#1}{null}}{\!}{{\sf(\bracketspace{#1})}}
	 \ifthenelse{\equal{#2}{null}}{\!\!}{{\sf [\color{Gray444}\,\bracketspace{{\fontfamily{afd}\selectfont#2}}\,]}}}.}

\DeclareDocumentCommand{\Pr}{ O{null} O{null} }{\setcounter{subproblem}{0}
	\beforPRskip\noindent\stepcounter{problem}\setcounter{prvar}{0}{\bf\theproblem
	~\!\!\! \ifthenelse{\equal{#1}{null}}{\!\!}{{\sf(\bracketspace{#1})}}
	 \ifthenelse{\equal{#2}{null}}{\!\!}{{\sf [\color{Gray444}\,\bracketspace{{\fontfamily{afd}\selectfont#2}}\,]}}}.}

%\DeclareDocumentCommand{\Prp}{ O{level} O{meta} }

\DeclareDocumentCommand{\Prps}{ O{null} O{null} }{\setcounter{subproblem}{0}
	\beforPRskip\noindent\stepcounter{problem}\setcounter{prvar}{0}{\bf Задача $\bm\theproblem^* $
	~\!\!\! \ifthenelse{\equal{#1}{null}}{\!}{{\sf(\bracketspace{#1})}}
	 \ifthenelse{\equal{#2}{null}}{\!\!}{{\sf [\color{Gray444}\,\bracketspace{{\fontfamily{afd}\selectfont#2}}\,]}}}.
}

\DeclareDocumentCommand{\Prpd}{ O{null} O{null} }{\setcounter{subproblem}{0}
	\beforPRskip\noindent\stepcounter{problem}\setcounter{prvar}{0}{\bf Задача $\bm\theproblem^\dagger$
	~\!\!\! \ifthenelse{\equal{#1}{null}}{\!}{{\sf(\bracketspace{#1})}}
	 \ifthenelse{\equal{#2}{null}}{\!\!}{{\sf [\color{Gray444}\,\bracketspace{{\fontfamily{afd}\selectfont#2}}\,]}}}.
}


\def\prend{
	\bigskip
%	\bigskip
}




%%%%%%%%%%%%%%%%%%%%%%%%%%%%%%% EOF Problems macros  %%%%%%%%%%%%%%%%%%%%%%%%%%%%%%%



%\usepackage{erewhon}
%\usepackage{heuristica}
%\usepackage{gentium}

\usepackage[portrait, top=3cm, bottom=1.5cm, left=3cm, right=2cm]{geometry}

\usepackage{fancyhdr}
\pagestyle{fancy}
\renewcommand{\headrulewidth}{0pt}
\lhead{\fontfamily{fca}\selectfont {Основные алгоритмы 2022} }
%\lhead{ \bf  {ТРЯП. } Семинар 1 }
%\chead{\fontfamily{fca}\selectfont {Вариант 1}}
\rhead{\fontfamily{fca}\selectfont Домашнее задание 4}
\rhead{\small Павлов М.А. 03.2022}
\cfoot{}

\usepackage{titlesec}
\titleformat{\section}[block]{\Large\bfseries\filcenter {\setcounter{problem}{0}}  }{}{1em}{}


%%%%%%%%%%%%%%%%%%%%%%%%%%%%%%%%%%%%%%%%%%%%%%%%%%%% Обозначения и операции %%%%%%%%%%%%%%%%%%%%%%%%%%%%%%%%%%%%%%%%%%%%%%%%%%%% 
                                                                    
\newcommand{\divisible}{\mathop{\raisebox{-2pt}{\vdots}}}           
\let\Om\Omega


%%%%%%%%%%%%%%%%%%%%%%%%%%%%%%%%%%%%%%%% Shen Macroses %%%%%%%%%%%%%%%%%%%%%%%%%%%%%%%%%%%%%%%%
\newcommand{\w}[1]{{\hbox{\texttt{#1}}}}


\begin{document}
	


\pr Дан массив длины $n$, состоящий только из нулей и единиц. Предложите линейный алгоритм сортировки данного массива.


\textbf{1) Заводим переменные счетчики кол-ва нулей $zcounter$.}

\textbf{2) При встрече очередного $0$ записываем в ячейку с индексом $zcounter$ значение $'0'$ и инкрементируем счетчик.}

\textbf{3) Когда прошлись по всему массиву, все ячейки, начиная с $zcounter$, заполняем единицами.}

\pr На прямой задано $n$ отрезков, причем известно, что они образуют систему строго вложенных отрезков (их можно упорядочить так, чтобы каждый строго содержался в следующем). Отрезки заданы координатами концов $[l_i, r_i]$ (и могут быть даны в неупорядоченном виде). Предложите асимптотически эффективный алгоритм (с точки зрения количества арифметических операций), который находит все точки прямой, которые покрыты ровно $2n/3$ отрезками.

\textbf{1) Находим порядковую статистику с номерами $\frac{2n}{3}$ и $\frac{2n}{3} + 1$. Поиск такой порядковой статистики пройдет за линейное время}

\textbf{2) Найденные промежутки (с выколотыми ближе к центру точками) по сути и будут искомыми, то есть мы решили задачу за линейное время!}

\pr Рассмотрим детерминированный алгоритм поиска порядковой статистики за линейное время из параграфа 9.3 Кормена. Какая асимптотика будет у алгоритма, если делить элементы массива на группы по семь, а не по пять?

\textbf{1) Разбиваем массив на блоки по 7 -- также за линейное время}

\textbf{2) Чтобы найти медиану в массиве медиан, нам понадобится $T(\frac{n}{7})$ операций}

\textbf{3) Далее определяем место, где будет находиться медиана медиан: $\frac{2n}{7} \leq med \leq \frac{5n}{7}$}

\textbf{4) В таком случае сложность алгоритма будет похожа на ту, что была получена в википедии: $T(n) = T\frac{5n}{7} + T(\frac{n}{7}) + cn$. Тогда $T(n) = \Theta(n)$}

\pr На вход задачи подаётся число $n$ и массив чисел $x_1, x_2, \ldots, x_{2n+1}$. Постройте линейный алгоритм, находящий число $s$, при котором достигается минимум суммы $$ \sum\limits_{i=1}^{2n+1}|x_i - s|. $$

\textbf{Чтобы найти это число $s$, необходимо найти "центр тяжести" всех точек, расположенных на прямой с координатами $x_i$.}

\textbf{В одной из задач выше мы находили медиану за линейное время. Здесь по сути нам тоже нужно воспользоваться алгоритмом поиска порядковой статистики за линейное время.}

\textbf{С помощью этого алгоритма мы найдем медиану, которая и будет являться искомой $s$.}

\pr Предложите полиномиальный от длины входа алгоритм решения сравнения $a\cdot x + b\equiv 0\pmod M$ (На вход дают целые числа $a,b,M$ в двоичной системе исчисления).

\textbf{Раньше мы сталкивались с похожими уравнениями, поэтому знаем, что $ax + b \equiv 0 (mod M)$ можно представить в виде $ax + b = M \cdot k$, где $k \in Z$ -- какое-то целое число.}

\textbf{С помощью расширенного алгоритма Евклида находим решения, которые будут равны $x = x_0 + \frac{M}{\text{НОД}}q, q \in Z$. }

\textbf{Таким образом, сложность алгоритма будет зависеть исключительно от алгоритма Евклида, который работает за полиномиальное время. }

\pr Перемножьте многочлены $2x^3 + 3x^2 + 1$ и $2x^2 + x$ с помощью БПФ. В решении должны быть приведены вычисления всех используемых преобразований.

$A(x) = 2x^3 + 3x^2 + 1$

$B(x) = 2x^2 + x$

$A(x)B(x) = P_5(n)$

\begin{table}[]
	\begin{tabular}{|l|l|l|l|l|l|l|l|}
		\hline
		1                       & 0                    & 3     & 2                    & 0 & 0                    & 0     & 0                    \\ \hline
		1 						& 3                    & 0     & 0                    & 0 & 2                    & 0     & 0                    \\ \hline
		1                       & 0                    & 3     & 0                    & 0 & 0                    & 2     & 0                    \\ \hline
		1                       & 1                    & 3     & 3                    & 0 & 0                    & 2     & 2                    \\ \hline
		4                       & 1+3i                 & -2    & 1-3i                 & 2 & 2i                   & -2    & -2i                  \\ \hline
		6                       & 1+3i+$\sqrt{2}$(i-1) & -2-2i & 1-3i-$\sqrt{2}$(i-1) & 2 & 1+3i-$\sqrt{2}$(i-1) & -2+2i & 1-3i+$\sqrt{2}$(i-1) \\ \hline
	\end{tabular}
	\caption{Быстрое преобразование Фурье}
	\label{tab:my-table}
\end{table}

\pr Решите с помощью преобразования Фурье задачу о поиске всех вхождений образца с джокерами в текст. Текст и образец~"--- это последовательности $t_0,t_1,\ldots, t_{n-1}$ и $p_0, p_1, \ldots, p_{m-1}$, $m < n$, где все $t_i$~"--- символы из алфавита, а $p_j$~"--- либо символ из алфавита, либо джокер. Образец входит в текст в позиции~$i \in \{0,\ldots, n-m-1\}$, если $t_{i+j} = p_j$ при всех $j \in \{0,\ldots, m-1\}$, для которых $p_j$~"--- символ алфавита. Для решения этой (и более сложной задачи в домашнем задании) есть $O(n\log n)$ алгоритм, основанный на БПФ. Закодируем каждый символ алфавита уникальным положительным числом, а джокер нулём, и определим последовательность $r_i$:
$r_i =  \sum_{j=0}^{m-1} p_{j} t_{i+j}\left(p_{j}-t_{i+j}\right)^{2}$

\prsub Докажите, что образец входит в текст в позиции~$i$ тогда и только тогда, когда $r_i = 0$.
 
\prsub Постройте $O(n\log n)$ алгоритм, который находит все вхождения образца с джокерами в текст.

\medskip

\noindent\textbf{Заметка.} Эта задача подготовлена на основе статьи~\href{https://www.cs.cmu.edu/afs/cs/academic/class/15750-s16/Handouts/WildCards2006.pdf}{P.~Clifford, R.~Clifford Simple deterministic wildcard matching, Information Processing Letters, Vol. 101, Is. 2, 2007, Pp. 53-54,}


\Pr[null][ДПВ \textbf{2.30}]
В данном упражнении показывается, как вычислять преобразование Фурье (ПФ) в арифметике сравнений, например, по модулю 7.

\prsub Существует такое $\omega$, что все степени $\omega, \omega^2,\ldots, \omega^6$ различны (по модулю 7). Найдите такое~$\omega$ и покажите, что $\omega + \omega^2 + \ldots  + \omega^6 = 0$. (Отметим также, что такое число существует для любого простого модуля.)

Эта $\omega$ равна 3, т.к.

$3^6 = 729 \equiv 1 mod 7$

$3^2 = 9 \equiv 2 mod 7$

$3^1 = 3 \equiv 3 mod 7$

$3^4 = 81 \equiv 4 mod 7$

$3^5 = 243 \equiv 5 mod 7$

$3^3 = 27 \equiv 6 mod 7$

$\omega + \omega^2 + \dots + \omega^6 \equiv 0 mod 7$ -- это следует непосредственно из того, что сумма остатков (21) делится на 7.

\prsub Найдите преобразование Фурье вектора $(0, 1, 1, 1, 5, 2)$ по модулю 7, используя матричное представление, то есть умножьте данный вектор на $M_6(\omega)$ (для найденного ранее $\omega$). Все промежуточные вычисления производите по модулю 7.

\textbf{Это будет $$ }

\prsub Запишите матрицу обратного преобразования Фурье. Покажите, что при умножении на эту матрицу получается исходный вектор. (Как и прежде, все вычисления должны производиться по модулю 7.)

\prsub Перемножьте многочлены $x^2 + x +1$ и $x^3 +2x -1$ при помощи ПФ по модулю 7.

\end{document}
  