\documentclass[12pt]{extreport}
\usepackage[T2A]{fontenc}
\usepackage[utf8]{inputenc}        % Кодировка входного документа;
                                    % при необходимости, вместо cp1251
                                    % можно указать cp866 (Alt-кодировка
                                    % DOS) или koi8-r.

\usepackage[english,russian]{babel} % Включение русификации, русских и
                                    % английских стилей и переносов
%%\usepackage{a4}
%%\usepackage{moreverb}
\usepackage{amsmath,amsfonts,amsthm,amssymb,amsbsy,amstext,amscd,amsxtra,multicol}
\usepackage{indentfirst}
\usepackage{verbatim}
\usepackage{tikz} %Рисование автоматов
\usetikzlibrary{automata,positioning}
\usepackage{multicol} %Несколько колонок
\usepackage{graphicx}
\usepackage[colorlinks,urlcolor=blue]{hyperref}
\usepackage[stable]{footmisc}

%% \voffset-5mm
%% \def\baselinestretch{1.44}
\renewcommand{\theequation}{\arabic{equation}}
\def\hm#1{#1\nobreak\discretionary{}{\hbox{$#1$}}{}}
\newtheorem{Lemma}{Лемма}
\theoremstyle{definiton}
\newtheorem{Remark}{Замечание}
%%\newtheorem{Def}{Определение}
\newtheorem{Claim}{Утверждение}
\newtheorem{Cor}{Следствие}
\newtheorem{Theorem}{Теорема}
\theoremstyle{definition}
\newtheorem{Example}{Пример}
\newtheorem*{known}{Теорема}
\def\proofname{Доказательство}
\theoremstyle{definition}
\newtheorem{Def}{Определение}

%% \newenvironment{Example} % имя окружения
%% {\par\noindent{\bf Пример.}} % команды для \begin
%% {\hfill$\scriptstyle\qed$} % команды для \end






%\date{22 июня 2011 г.}
\let\leq\leqslant
\let\geq\geqslant
\def\MT{\mathrm{MT}}
%Обозначения ``ажуром''
\def\BB{\mathbb B}
\def\CC{\mathbb C}
\def\RR{\mathbb R}
\def\SS{\mathbb S}
\def\ZZ{\mathbb Z}
\def\NN{\mathbb N}
\def\FF{\mathbb F}
%греческие буквы
\let\epsilon\varepsilon
\let\es\varnothing
\let\eps\varepsilon
\let\al\alpha
\let\sg\sigma
\let\ga\gamma
\let\ph\varphi
\let\om\omega
\let\ld\lambda
\let\Ld\Lambda
\let\vk\varkappa
\let\Om\Omega
\def\abstractname{}

\def\R{{\cal R}}
\def\A{{\cal A}}
\def\B{{\cal B}}
\def\C{{\cal C}}
\def\D{{\cal D}}

%классы сложности
\def\REG{{\mathsf{REG}}}
\def\CFL{{\mathsf{CFL}}}


%%%%%%%%%%%%%%%%%%%%%%%%%%%%%%% Problems macros  %%%%%%%%%%%%%%%%%%%%%%%%%%%%%%%


%%%%%%%%%%%%%%%%%%%%%%%% Enumerations %%%%%%%%%%%%%%%%%%%%%%%%

\newcommand{\Rnum}[1]{\expandafter{\romannumeral #1\relax}}
\newcommand{\RNum}[1]{\uppercase\expandafter{\romannumeral #1\relax}}

%%%%%%%%%%%%%%%%%%%%% EOF Enumerations %%%%%%%%%%%%%%%%%%%%%

\usepackage{xparse}
\usepackage{ifthen}
\usepackage{bm} %%% bf in math mode
\usepackage{color}
%\usepackage[usenames,dvipsnames]{xcolor}

\definecolor{Gray555}{HTML}{555555}
\definecolor{Gray444}{HTML}{444444}
\definecolor{Gray333}{HTML}{333333}


\newcounter{problem}
\newcounter{uproblem}
\newcounter{subproblem}
\newcounter{prvar}

\def\beforPRskip{
	\bigskip
	%\vspace*{2ex}
}

\def\PRSUBskip{
	\medskip
}


\def\pr{\beforPRskip\noindent\stepcounter{problem}{\bf \theproblem .\;}\setcounter{subproblem}{0}}
\def\pru{\beforPRskip\noindent\stepcounter{problem}{\bf $\mathbf{\theproblem}^\circ$\!\!.\;}\setcounter{subproblem}{0}}
\def\prstar{\beforPRskip\noindent\stepcounter{problem}{\bf $\mathbf{\theproblem}^*$\negthickspace.}\setcounter{subproblem}{0}\;}
\def\prpfrom[#1]{\beforPRskip\noindent\stepcounter{problem}{\bf Задача \theproblem~(№#1 из задания).  }\setcounter{subproblem}{0} }
\def\prp{\beforPRskip\noindent\stepcounter{problem}{\bf Задача \theproblem .  }\setcounter{subproblem}{0} }

\def\prpvar{\beforPRskip\noindent\stepcounter{problem}\setcounter{prvar}{1}{\bf Задача \theproblem \;$\langle${\rm\Rnum{\theprvar}}$\rangle$.}\setcounter{subproblem}{0}\;}
\def\prpv{\beforPRskip\noindent\stepcounter{prvar}{\bf Задача \theproblem \,$\bm\langle$\bracketspace{{\rm\Rnum{\theprvar}}}$\bm\rangle$.  }\setcounter{subproblem}{0} }
\def\prv{\beforPRskip\noindent\stepcounter{prvar}{\bf \theproblem\,$\bm\langle$\bracketspace{{\rm\Rnum{\theprvar}}}$\bm\rangle$}.\setcounter{subproblem}{0} }

\def\prpstar{\beforPRskip\noindent\stepcounter{problem}{\bf Задача $\bf\theproblem^*$\negthickspace.  }\setcounter{subproblem}{0} }
\def\prdag{\beforPRskip\noindent\stepcounter{problem}{\bf Задача $\theproblem^{^\dagger}$\negthickspace\,.  }\setcounter{subproblem}{0} }
\def\upr{\beforPRskip\noindent\stepcounter{uproblem}{\bf Упражнение \theuproblem .  }\setcounter{subproblem}{0} }
%\def\prp{\vspace{5pt}\stepcounter{problem}{\bf Задача \theproblem .  } }
%\def\prs{\vspace{5pt}\stepcounter{problem}{\bf \theproblem .*   }
\def\prsub{\PRSUBskip\noindent\stepcounter{subproblem}{\sf \thesubproblem .} }
\def\prsubr{\PRSUBskip\noindent\stepcounter{subproblem}{\bf \asbuk{subproblem})}\;}
\def\prsubstar{\PRSUBskip\noindent\stepcounter{subproblem}{\rm $\thesubproblem^*$\negthickspace.  } }
\def\prsubrstar{\PRSUBskip\noindent\stepcounter{subproblem}{$\text{\bf \asbuk{subproblem}}^*\mathbf{)}$}\;}

\newcommand{\bracketspace}[1]{\phantom{(}\!\!{#1}\!\!\phantom{)}}

\DeclareDocumentCommand{\Prpvar}{ O{null} O{} }{
	\beforPRskip\noindent\stepcounter{problem}\setcounter{prvar}{1}{\bf Задача \theproblem
% 	\ifthenelse{\equal{#1}{null}}{  }{ {\sf $\bm\langle$\bracketspace{#1}$\bm\rangle$}}
%	~\!\!(\bracketspace{{\rm\Rnum{\theprvar}}}).  }\setcounter{subproblem}{0}
%	\;(\bracketspace{{\rm\Rnum{\theprvar}}})}\setcounter{subproblem}{0}
%
	\,{\sf $\bm\langle$\bracketspace{{\rm\Rnum{\theprvar}}}$\bm\rangle$}
	~\!\!\! \ifthenelse{\equal{#1}{null}}{\!}{{\sf(\bracketspace{#1})}}}.

}
%\DeclareDocumentCommand{\Prpvar}{ O{level} O{meta} m }{\prpvar}


\DeclareDocumentCommand{\Prp}{ O{null} O{null} }{\setcounter{subproblem}{0}
	\beforPRskip\noindent\stepcounter{problem}\setcounter{prvar}{0}{\bf Задача \theproblem
	~\!\!\! \ifthenelse{\equal{#1}{null}}{\!}{{\sf(\bracketspace{#1})}}
	 \ifthenelse{\equal{#2}{null}}{\!\!}{{\sf [\color{Gray444}\,\bracketspace{{\fontfamily{afd}\selectfont#2}}\,]}}}.}

\DeclareDocumentCommand{\Pr}{ O{null} O{null} }{\setcounter{subproblem}{0}
	\beforPRskip\noindent\stepcounter{problem}\setcounter{prvar}{0}{\bf\theproblem
	~\!\!\! \ifthenelse{\equal{#1}{null}}{\!\!}{{\sf(\bracketspace{#1})}}
	 \ifthenelse{\equal{#2}{null}}{\!\!}{{\sf [\color{Gray444}\,\bracketspace{{\fontfamily{afd}\selectfont#2}}\,]}}}.}

%\DeclareDocumentCommand{\Prp}{ O{level} O{meta} }

\DeclareDocumentCommand{\Prps}{ O{null} O{null} }{\setcounter{subproblem}{0}
	\beforPRskip\noindent\stepcounter{problem}\setcounter{prvar}{0}{\bf Задача $\bm\theproblem^* $
	~\!\!\! \ifthenelse{\equal{#1}{null}}{\!}{{\sf(\bracketspace{#1})}}
	 \ifthenelse{\equal{#2}{null}}{\!\!}{{\sf [\color{Gray444}\,\bracketspace{{\fontfamily{afd}\selectfont#2}}\,]}}}.
}

\DeclareDocumentCommand{\Prpd}{ O{null} O{null} }{\setcounter{subproblem}{0}
	\beforPRskip\noindent\stepcounter{problem}\setcounter{prvar}{0}{\bf Задача $\bm\theproblem^\dagger$
	~\!\!\! \ifthenelse{\equal{#1}{null}}{\!}{{\sf(\bracketspace{#1})}}
	 \ifthenelse{\equal{#2}{null}}{\!\!}{{\sf [\color{Gray444}\,\bracketspace{{\fontfamily{afd}\selectfont#2}}\,]}}}.
}


\def\prend{
	\bigskip
%	\bigskip
}




%%%%%%%%%%%%%%%%%%%%%%%%%%%%%%% EOF Problems macros  %%%%%%%%%%%%%%%%%%%%%%%%%%%%%%%



%\usepackage{erewhon}
%\usepackage{heuristica}
%\usepackage{gentium}

\usepackage[portrait, top=3cm, bottom=1.5cm, left=3cm, right=2cm]{geometry}

\usepackage{fancyhdr}
\pagestyle{fancy}
\renewcommand{\headrulewidth}{0pt}
\lhead{\fontfamily{fca}\selectfont {Основные алгоритмы 2020} }
%\lhead{ \bf  {ТРЯП. } Семинар 1 }
%\chead{\fontfamily{fca}\selectfont {Вариант 1}}
\rhead{\fontfamily{fca}\selectfont Домашнее задание 1}
%\rhead{\small 01.09.2016}
\cfoot{}

\usepackage{titlesec}
\usepackage{lipsum}
\titleformat{\section}[block]{\Large\bfseries\filcenter {\setcounter{problem}{0}}  }{}{1em}{}


%%%%%%%%%%%%%%%%%%%%%%%%%%%%%%%%%%%%%%%%%%%%%%%%%%%% Обозначения и операции %%%%%%%%%%%%%%%%%%%%%%%%%%%%%%%%%%%%%%%%%%%%%%%%%%%% 
                                                                    
\newcommand{\divisible}{\mathop{\raisebox{-2pt}{\vdots}}}           
\let\Om\Omega


%%%%%%%%%%%%%%%%%%%%%%%%%%%%%%%%%%%%%%%% Shen Macroses %%%%%%%%%%%%%%%%%%%%%%%%%%%%%%%%%%%%%%%%
\newcommand{\w}[1]{{\hbox{\texttt{#1}}}}


\begin{document}

\pr Высота дерева $h = log_4 n, T(N) = 3T(\frac{n}{4} + C \Rightarrow a = 3, b = 4 \Rightarrow d = log_4 3)$

	$f(n) = \Theta(n^{log_4 3})$

	$C = O(n^{log_4 3 - \varepsilon}) \Rightarrow T(n) = \Theta (n^{log_4 3})$

\pr	Пусть сумма всех записанных на доске чисел изначально равна $S$. Поскольку на каждом шаге мы берем два числа $a$ и $b$ ($a > b$) и
вычитаем из $a b$, то сумма $S = S - (a-b)$, то есть $S(n)$ -- монотонно убывающая функция. Т.к. $S$ монотонно убывает и ограничена снизу (т.к. все числа на доске положительны), то,
очевидно, процесс когда-нибудь остановится, т.к. $S$ обязательно достигнет значения $S_{min}$.

Оставшиеся числа будут равны $d = $НОД$(x_1\dotsx_n)$), т.к. разность любых двух чисел кратна $d$ и сами числа кратны $d$.

\pr 1) Складываем два числа -- сложность $O(n)$

2) Возводим результат в квадрат -- сложнсть $O(m + n) = O(2n) = O(n)$

3) Находим квадраты первого и второго чисел -- сложность $O(m)$ и $O(n) = O(n)$

4) Вычитаем из (2) квадраты -- сложность $O(m+n) = O(n)$

5) Делим результат на 2 -- сложность $O(n)$

В итоге получаем сложность $O(n)$

\pr Самый простой способ нахождения НОК через НОД с помощью формулы: НОК($a, b$) = $\frac{a \cdot b }{\text{НОД}(a, b)}$.

Для получения НОД существует алгоритм Евклида, который работает за $\Theta(n^2)$ (за счет операции деления с остатком).

Далее нам остается произвести операции умножения и деления, которые также совершаются за $\Theta(n^2)$.

Таким образом, суммарная сложность алгоритма -- $\Theta (n^2)$.

\pr Еще из школы мы знаем формулу $(a_1 + a_2 + \dots + a_n)^2 = a_1^2 + a_2^2 + \dots + a_n^2$ и плюс все попарные произведения.

Запишем алгоритм:

1) Вычисляем сумму всех чисел -- $O(n)$

2) Вычисляем сумму квадратов всех чисел -- $O(n)$

3) Вычисляем квадрат суммы всех чисел -- $O(1)$

4) Вычисляем полуразность (3) и (2) -- $O(1)$

Полученный результат будет равен искомой сумме. Таким образом, мы нашли значение необходимой суммы за $O(n)$.

\pr В этой задаче, очевидно, стоит использовать $master$ теорему.

\textbf{а)} $a = 36, b = 6 \Rightarrow log_b a = 2 \Rightarrow F(n) = \Theta(n^2)$

$f(n) = n^2 = \Theta(n^2)$

В таком случае получаем ответ $T(n) = \Theta(n^2logn)$

\textbf{б)} $log_b a = 1 \Rightarrow F(n) = \Theta(n)$

$f(n) = n^2 = \Omega(n^{log_b a + \varepsilon})$

В таком случае $T(n) = \Theta (n^2)$

\textbf{в)} $log_b a = 2 \Rightarrow F(n) = \Theta(n)$

$f(n) = \frac{n}{logn} = O(n^{log_b a - \varepsilon})$

В таком случае $T(n) = \Theta (n^2)$

\pr По сути нам достаточно просто применить к массиву сортировку слиянием, считая каждую перестановку.
(Соответственно если мы переставили сразу несколько элементов, то мы прибавляем к счетчику число этих элементов)
$\sum\limits{1}{k}n_i$ (где $k$ - количество перестановок, а $n_i$ -- количество элементов которые переставляют в $i-$той перестановке) будет равна искомому количеству инверсий.

\pr Воспользуемся мастер-теооремой. Здесь будет 3 случая:

1) Начнем с того, что $aT_1\frac{n}{b} = \Theta(n^{log_b a}) = aT_2\frac{n}{b}$

2) Из (1) и из $f(n) = \Theta(g(n)) \Rightarrow T_1(n) = \Theta(T_2(n))$.


{Задачу можно решать считая, что их нет, т.е. считать, что на вход $T(n)$ могут подаваться дробные параметры. Это облегчит задачу. Однако для того, чтобы лучше разобраться в теме, рекомендуем решать задачи с учётом округлений.}

\pr Найдите $\Theta$-асимптотику рекуррентной последовательности $T(n)$, считая что $T(n)$ ограничено константой при достаточно малых $n$:

\textbf{В этой задаче везде используется master-теорема, так что высказывания о том, что я использую ее, я, пожалуй, опущу}

\prsubr $T(n) = 3T(\lfloor {n}/4\rfloor) + T(\lceil {n}/6\rceil) + n$;

$T(n) = \Theta(n^{log_4 3}) + \Theta(n^{log_6 1}) + \Omega(n^{log_4 3 + \varepsilon}) = \Theta(f(n)) = \Theta(n)$

\prsubr $T(n)=T(\lfloor \alpha n\rfloor)+T( \lfloor(1-\alpha)n\rfloor)+\Theta(n)$\quad  ($0 < \alpha < 1$);

В силу симметрии и коммутативности суммы можем считать $\alpha < \frac{1}{2}$

В таком случае длина самой короткой ветви будет равна $log_{\alpha}n$. Длина самой длиной ветви -- $log_{1 - \alpha}n.$

Тогда справедливы неравенства $\log_{\alpha}n \cdot \Theta(n) \leq T(n) \leq log_{1-\alpha}n \cdot \Theta(n)$, т.е. $T(n) = \Theta(nlogn)$

\prsubr $T(n)=T(\lfloor n/2\rfloor)+2\cdot T(\lfloor n/4\rfloor)+\Theta(n)$;

Для $T(\frac{n}{2})$ сложность $O(n^{log_2 1})$. Для $2T(\frac{n}{4})$ сложность $O(n^{log_4 2})$.

Т.к. $f(n) = \Theta(n) = \Omega(n^{log_4 2 + \varepsilon_1} = \Omega(n^{log_2 1 + \varepsilon_2}))$, то $T(n) = \Theta(nlogn)$.

\prsubr $T(n) = 27T(\frac{n}{3})+\frac{n^3}{\log^2 n} $.

Для $27T(\frac{n}{3})$ имеем $n^{log_3 27} = n^{3}$.

Т.к. $f(n) = O(n^{3 - \varepsilon})$, то $T(n) = \Theta(n^3)$.
                                                                                                              
\pr На вход подаются натуральные числа $n, p$, $n < p, p$ -- простое. Предложите алгоритм, который за $O(n + \log p)$ арифметических операций вычисляет массив длины $n$ ($i$ пробегает значение от $1$ до $n$):

\prsubr $\w{invfac[}i\w{]} = (i!)^{-1}\pmod p$;



\prsubrstar $\w{inv[}i\w{]} = (i)^{-1}\pmod p$.

\prstar Оцените трудоемкость рекурсивного алгоритма, разбивающего исходную задачу размера $n$ на $n$ задач размеров $\lceil \frac n 2 \rceil$ каждая, используя для этого $\Theta(n)$ операций. 

\prsub Можно считать $n$ степенью двойки.                

Здесь мы имеем $T(n) = nT(\frac{n}{2})$.

Высота дерева равна $log_2 n$.

Количество вызовов на $i-$том уровне: $n(\frac{n}{2})^i$

В итоге имеем $n \cdot \sum\limits_{0}^{log_2 n} (\frac{n}{2})^i = \Theta (n \cdot \frac{n^{log_2 n}}{2^{log_2n}}) = \Theta (n^{log_2n})$.

\prsub Решите для произвольного $n$.

	Аналогично предыдущему пункту, но при делении на 2 просто всегда округляем вверх.

\end{document}
  