\documentclass[12pt]{extreport}
\usepackage[T2A]{fontenc}
\usepackage[utf8]{inputenc}        % Кодировка входного документа;
% при необходимости, вместо cp1251
% можно указать cp866 (Alt-кодировка
% DOS) или koi8-r.

\usepackage[english,russian]{babel} % Включение русификации, русских и
% английских стилей и переносов
%%\usepackage{a4}
%%\usepackage{moreverb}
\usepackage{amsmath,amsfonts,amsthm,amssymb,amsbsy,amstext,amscd,amsxtra,multicol}
\usepackage{indentfirst}
\usepackage{verbatim}
\usepackage{tikz} %Рисование автоматов
\usetikzlibrary{automata,positioning}
\usepackage{multicol} %Несколько колонок
\usepackage{graphicx}
\usepackage[colorlinks,urlcolor=blue]{hyperref}
\usepackage[stable]{footmisc}

%% \voffset-5mm
%% \def\baselinestretch{1.44}
\renewcommand{\theequation}{\arabic{equation}}
\def\hm#1{#1\nobreak\discretionary{}{\hbox{$#1$}}{}}
\newtheorem{Lemma}{Лемма}
\theoremstyle{definiton}
\newtheorem{Remark}{Замечание}
%%\newtheorem{Def}{Определение}
\newtheorem{Claim}{Утверждение}
\newtheorem{Cor}{Следствие}
\newtheorem{Theorem}{Теорема}
\theoremstyle{definition}
\newtheorem{Example}{Пример}
\newtheorem*{known}{Теорема}
\def\proofname{Доказательство}
\theoremstyle{definition}
\newtheorem{Def}{Определение}

%% \newenvironment{Example} % имя окружения
%% {\par\noindent{\bf Пример.}} % команды для \begin
%% {\hfill$\scriptstyle\qed$} % команды для \end






%\date{22 июня 2011 г.}
\let\leq\leqslant
\let\geq\geqslant
\def\MT{\mathrm{MT}}
%Обозначения ``ажуром''
\def\BB{\mathbb B}
\def\CC{\mathbb C}
\def\RR{\mathbb R}
\def\SS{\mathbb S}
\def\ZZ{\mathbb Z}
\def\NN{\mathbb N}
\def\FF{\mathbb F}
%греческие буквы
\let\epsilon\varepsilon
\let\es\varnothing
\let\eps\varepsilon
\let\al\alpha
\let\sg\sigma
\let\ga\gamma
\let\ph\varphi
\let\om\omega
\let\ld\lambda
\let\Ld\Lambda
\let\vk\varkappa
\let\Om\Omega
\def\abstractname{}

\def\R{{\cal R}}
\def\A{{\cal A}}
\def\B{{\cal B}}
\def\C{{\cal C}}
\def\D{{\cal D}}

%классы сложности
\def\REG{{\mathsf{REG}}}
\def\CFL{{\mathsf{CFL}}}


%%%%%%%%%%%%%%%%%%%%%%%%%%%%%%% Problems macros  %%%%%%%%%%%%%%%%%%%%%%%%%%%%%%%


%%%%%%%%%%%%%%%%%%%%%%%% Enumerations %%%%%%%%%%%%%%%%%%%%%%%%

\newcommand{\Rnum}[1]{\expandafter{\romannumeral #1\relax}}
\newcommand{\RNum}[1]{\uppercase\expandafter{\romannumeral #1\relax}}

%%%%%%%%%%%%%%%%%%%%% EOF Enumerations %%%%%%%%%%%%%%%%%%%%%

\usepackage{xparse}
\usepackage{ifthen}
\usepackage{bm} %%% bf in math mode
\usepackage{color}
%\usepackage[usenames,dvipsnames]{xcolor}

\definecolor{Gray555}{HTML}{555555}
\definecolor{Gray444}{HTML}{444444}
\definecolor{Gray333}{HTML}{333333}


\newcounter{problem}
\newcounter{uproblem}
\newcounter{subproblem}
\newcounter{prvar}

\def\beforPRskip{
    \bigskip
%\vspace*{2ex}
}

\def\PRSUBskip{
    \medskip
}


\def\pr{\beforPRskip\noindent\stepcounter{problem}{\bf \theproblem .\;}\setcounter{subproblem}{0}}
\def\pru{\beforPRskip\noindent\stepcounter{problem}{\bf $\mathbf{\theproblem}^\circ$\!\!.\;}\setcounter{subproblem}{0}}
\def\prstar{\beforPRskip\noindent\stepcounter{problem}{\bf $\mathbf{\theproblem}^*$\negthickspace.}\setcounter{subproblem}{0}\;}
\def\prpfrom[#1]{\beforPRskip\noindent\stepcounter{problem}{\bf Задача \theproblem~(№#1 из задания).  }\setcounter{subproblem}{0} }
\def\prp{\beforPRskip\noindent\stepcounter{problem}{\bf Задача \theproblem .  }\setcounter{subproblem}{0} }

\def\prpvar{\beforPRskip\noindent\stepcounter{problem}\setcounter{prvar}{1}{\bf Задача \theproblem \;$\langle${\rm\Rnum{\theprvar}}$\rangle$.}\setcounter{subproblem}{0}\;}
\def\prpv{\beforPRskip\noindent\stepcounter{prvar}{\bf Задача \theproblem \,$\bm\langle$\bracketspace{{\rm\Rnum{\theprvar}}}$\bm\rangle$.  }\setcounter{subproblem}{0} }
\def\prv{\beforPRskip\noindent\stepcounter{prvar}{\bf \theproblem\,$\bm\langle$\bracketspace{{\rm\Rnum{\theprvar}}}$\bm\rangle$}.\setcounter{subproblem}{0} }

\def\prpstar{\beforPRskip\noindent\stepcounter{problem}{\bf Задача $\bf\theproblem^*$\negthickspace.  }\setcounter{subproblem}{0} }
\def\prdag{\beforPRskip\noindent\stepcounter{problem}{\bf Задача $\theproblem^{^\dagger}$\negthickspace\,.  }\setcounter{subproblem}{0} }
\def\upr{\beforPRskip\noindent\stepcounter{uproblem}{\bf Упражнение \theuproblem .  }\setcounter{subproblem}{0} }
%\def\prp{\vspace{5pt}\stepcounter{problem}{\bf Задача \theproblem .  } }
%\def\prs{\vspace{5pt}\stepcounter{problem}{\bf \theproblem .*   }
\def\prsub{\PRSUBskip\noindent\stepcounter{subproblem}{\sf \thesubproblem .} }
\def\prsubr{\PRSUBskip\noindent\stepcounter{subproblem}{\bf \asbuk{subproblem})}\;}
\def\prsubstar{\PRSUBskip\noindent\stepcounter{subproblem}{\rm $\thesubproblem^*$\negthickspace.  } }
\def\prsubrstar{\PRSUBskip\noindent\stepcounter{subproblem}{$\text{\bf \asbuk{subproblem}}^*\mathbf{)}$}\;}

\newcommand{\bracketspace}[1]{\phantom{(}\!\!{#1}\!\!\phantom{)}}

\DeclareDocumentCommand{\Prpvar}{ O{null} O{} }{
    \beforPRskip\noindent\stepcounter{problem}\setcounter{prvar}{1}{\bf Задача \theproblem
% 	\ifthenelse{\equal{#1}{null}}{  }{ {\sf $\bm\langle$\bracketspace{#1}$\bm\rangle$}}
%	~\!\!(\bracketspace{{\rm\Rnum{\theprvar}}}).  }\setcounter{subproblem}{0}
%	\;(\bracketspace{{\rm\Rnum{\theprvar}}})}\setcounter{subproblem}{0}
%
    \,{\sf $\bm\langle$\bracketspace{{\rm\Rnum{\theprvar}}}$\bm\rangle$}
    ~\!\!\! \ifthenelse{\equal{#1}{null}}{\!}{{\sf(\bracketspace{#1})}}}.

}
%\DeclareDocumentCommand{\Prpvar}{ O{level} O{meta} m }{\prpvar}


\DeclareDocumentCommand{\Prp}{ O{null} O{null} }{\setcounter{subproblem}{0}
\beforPRskip\noindent\stepcounter{problem}\setcounter{prvar}{0}{\bf Задача \theproblem
~\!\!\! \ifthenelse{\equal{#1}{null}}{\!}{{\sf(\bracketspace{#1})}}
\ifthenelse{\equal{#2}{null}}{\!\!}{{\sf [\color{Gray444}\,\bracketspace{{\fontfamily{afd}\selectfont#2}}\,]}}}.}

\DeclareDocumentCommand{\Pr}{ O{null} O{null} }{\setcounter{subproblem}{0}
\beforPRskip\noindent\stepcounter{problem}\setcounter{prvar}{0}{\bf\theproblem
~\!\!\! \ifthenelse{\equal{#1}{null}}{\!\!}{{\sf(\bracketspace{#1})}}
\ifthenelse{\equal{#2}{null}}{\!\!}{{\sf [\color{Gray444}\,\bracketspace{{\fontfamily{afd}\selectfont#2}}\,]}}}.}

%\DeclareDocumentCommand{\Prp}{ O{level} O{meta} }

\DeclareDocumentCommand{\Prps}{ O{null} O{null} }{\setcounter{subproblem}{0}
\beforPRskip\noindent\stepcounter{problem}\setcounter{prvar}{0}{\bf Задача $\bm\theproblem^* $
    ~\!\!\! \ifthenelse{\equal{#1}{null}}{\!}{{\sf(\bracketspace{#1})}}
    \ifthenelse{\equal{#2}{null}}{\!\!}{{\sf [\color{Gray444}\,\bracketspace{{\fontfamily{afd}\selectfont#2}}\,]}}}.
}

\DeclareDocumentCommand{\Prpd}{ O{null} O{null} }{\setcounter{subproblem}{0}
\beforPRskip\noindent\stepcounter{problem}\setcounter{prvar}{0}{\bf Задача $\bm\theproblem^\dagger$
    ~\!\!\! \ifthenelse{\equal{#1}{null}}{\!}{{\sf(\bracketspace{#1})}}
    \ifthenelse{\equal{#2}{null}}{\!\!}{{\sf [\color{Gray444}\,\bracketspace{{\fontfamily{afd}\selectfont#2}}\,]}}}.
}


\def\prend{
    \bigskip
%	\bigskip
}




%%%%%%%%%%%%%%%%%%%%%%%%%%%%%%% EOF Problems macros  %%%%%%%%%%%%%%%%%%%%%%%%%%%%%%%



%\usepackage{erewhon}
%\usepackage{heuristica}
%\usepackage{gentium}

\usepackage[portrait, top=3cm, bottom=1.5cm, left=3cm, right=2cm]{geometry}

\usepackage{fancyhdr}
\pagestyle{fancy}
\renewcommand{\headrulewidth}{0pt}
\lhead{\fontfamily{fca}\selectfont {Основные алгоритмы 2022} }
%\lhead{ \bf  {ТРЯП. } Семинар 1 }
%\chead{\fontfamily{fca}\selectfont {Вариант 1}}
\rhead{\fontfamily{fca}\selectfont Домашнее задание 5}
\rhead{\small Павлов М.А. 03.2022}
\cfoot{}

\usepackage{titlesec}
\titleformat{\section}[block]{\Large\bfseries\filcenter {\setcounter{problem}{0}}  }{}{1em}{}


%%%%%%%%%%%%%%%%%%%%%%%%%%%%%%%%%%%%%%%%%%%%%%%%%%%% Обозначения и операции %%%%%%%%%%%%%%%%%%%%%%%%%%%%%%%%%%%%%%%%%%%%%%%%%%%%

\newcommand{\divisible}{\mathop{\raisebox{-2pt}{\vdots}}}
\let\Om\Omega


%%%%%%%%%%%%%%%%%%%%%%%%%%%%%%%%%%%%%%%% Shen Macroses %%%%%%%%%%%%%%%%%%%%%%%%%%%%%%%%%%%%%%%%
\newcommand{\w}[1]{{\hbox{\texttt{#1}}}}


% Document
\begin{document}

    \pr \textbf{24 числа}

        Количество перестановок -- $24!$.

        Составим условие для эквивалентной задачи: найти вероятность, что число $24$ находится среди первых 12-ти чисел.
        Эта задача действительно эквивалентна, поскольку число 24 в любом случае будет присутствовать либо среди первых 12 чисел,
        либо среди последних 12, и "победит" та половина, в которой есть число 24 (т.к. оно наибольшее при любом раскладе).

        Таким образом, искомая вероятность равна вероятности, что число 24 находится среди первых 12 чисел, т.е. $P = \frac{12}{24} = \frac{1}{2}$ -- это итоговый ответ.

    \pr \textbf{Чет-нечет}

        Среди чисел от 1 до 100 на 3 делится ровно 33, из которых 16 четных.

        Вероятность, что число делится на 2 при условии, что оно делится на 3, будет равна $\frac{16}{33}$

    \pr \textbf{Лото}

        Всего исходов -- $C_{36}^5$

        Рассчитаем вероятности:

        1) Среди выбранных чисел есть 2: $\frac{C_{35}^4}{C_{36}^5} = \frac{5}{36}$ (вероятность, что среди чисел есть 3 такая же).

        2) Среди выбранных чисел есть 3 при условии, что среди них есть 2: $\frac{C_{34}^3}{C_{35}^4} = \frac{4}{35}$.

        Т.к. $\frac{4}{35} \neq \frac{5}{36}$, то рассматриваемые события не являются независимыми.

    \pr \textbf{Всюду определенная функция}

        Если $f$ инъективна или если $f$ не инъективна, то событие $f(1) = x$ для всех $x$ равновероятно вне зависимости от первого события.

        Значит, рассматриваемые собылия не независимы.

    \pr \textbf{Коррумпированное жюри}

        Рассмотрим возможные вероятности при выборе решения первых двух членов жюри:

        1) Оба выбрали правильное: $p^2$.

        2) Один выбрал правильное, а другой -- неправильное: $2 \cdot p(1-p)$.

        3) Оба выбрали неправильное: $(1-p)^2$

        (Несложно убедиться, что сумма всех этих вот вероятностей равна 1, что не противоречит законам теории вероятности и элементарной логиики).

        Запишем формулу расчета вероятности с учетом третьего члена жюри:

        $p^2 \cdot 1 + 2 \cdot p(1-p) \cdot \frac{1}{2} + (1-p)^2 \cdot 0 = p^2 + p - p^2 = p$.

        Можно заметить, что эта вероятность совпадаем с вероятностью принятия правильного решения одним честным членом жюри.

    \pr \textbf{Казнить нельзя помиловать!}

        Вроде бы ответ очевиден, но нужно придумать доказательство\ldots

        Если $f$ -- вероятность вытащить белый шар из первой коробки, а $s$ -- вероятность вытащить белый шар из второй коробки, то
        вероятность вытащить белый шар будет равна $\frac{1}{2}f + \frac{1}{2}s = \frac{1}{2}(f + s)$.

        Таким образом, нам нужно добиться, чтобы сумма $f + s$ была максимальной.

        $f = \frac{f_n}{f_d}, s = \frac{s_n}{s_d}$, причем $f_d + s_d = 20$ и $f_n + s_n = 10$.

        Таким образом $f + s = \frac{f_n}{f_d} + \frac{10 - f_n}{20 - f_d} = \frac{f_n \cdot (20 - f_d) + f_d \cdot (10 - f_n)}{f_d \cdot (20 - f_d)}$.

        Если рассматривать $f = f(f_d)$, то при $f_d = 20$ или $f_d = 0$. (достаточно очевидно, чтобы не показывать вычисления)

        Таким образом, мы достигнем идеала (в смысле идеальной для нас ситуации), если в одной из коробок будет 10 белых шаров и 20 шаров всего, а во второй - 0 шаров.
        Но абсолютный идеал в условиях данной задачи недостижим, поэтому мы кладем 1 белый шар во вторую коробку, где он будет скучать в одиночестве.

        Тогда вероятность будет равна $\frac{1}{2} \cdot 1 + \frac{1}{2} \cdot \frac{9}{19} = \frac{14}{19}$ -- именно с такой вероятностью узника казнят!

    \pr \textbf{20-типартийный матч}

        Итак, у нас осталось как максимум 5 партий до конца игры, поэтому хочется сказать, что количество вариантов развития событий -- $2^5$,
        но, увы, это не так, ведь если первый игрок выиграет 2 партии, то он побеждает в матче (т.к. у него наберется 10 побед).
        И аналогично второй игрок побеждает, если выиграет 3 партии.

        Посчитаем общее количество исходов:

        1) Первый выиграл 0 партий: 1 исход

        2) Первый выиграл 1 партию (значит, второй выиграл 3): $C_3^1 = 3$ исходов.

        3) Первый выиграл 2 партии (второй выиграл не более двух): $1! + 2! + C_3^1 = 6$ исходов.

        Итого $\underline{10}$ исходов.

        Нас устраивает, когда первый выиграл 2 партии (6 исходов) $\Rightarrow P(W2) = \frac{6}{10}$ -- это ответ.

    \pr \textbf{Бой яиц}

        Назовем игроков по классике Петей (первый игрок) и Васей (второй игрок).

        Пусть Петя победил в первых $n$ раундах. В таком случае, если рассматривать первые $n+1$ яиц, то Петино яйцо самое крепкое из них.
        Значит, если Вася возьмет в $n+1$-ом раунде $n+2$-ое яйцо, то оно по прочности может занять с равной вероятностью (по условию задачи)
        любое из $n+2$-ух мест. Причем его яйцо будет прочнее Васиного только в том случае, если оно займет позицию по прочности на самом верху (то есть 1 благоприятный исход из $n+2$).
        Но мы вроде ищем вероятность победы Васи. Поскольку ничья невозможно, то кто-то из них победит с вероятностью 1, значит,
        Вася победит с вероятностью $1 - \frac{1}{n+2} = \frac{n + 1}{n + 2}$ -- это и будет ответом.

    \pr \textbf{Двоичные слова}

        Количество исходов всего -- $2^21$.
        Но нам это число не пригодится, потому что мы будем решать задачу из соображения симметрии.

        Заметим, что для любого набора нулей и единиц, который нам подходит, найдется набор (если инвертировать число), который нам не подходит.
        И аналогично для каждого набора, который нам не подходит, существует набор (если инвертировать число), который нам подходит.

        Таким образом, инвертирование числа биективно, откуда следует, что количества благоприятных и неблагоприятных исходов равны.

        Значит, искомая вероятность -- $\frac{1}{2}$.

\end{document}