\documentclass[12pt]{extreport}
\usepackage[T2A]{fontenc}
\usepackage[utf8]{inputenc}        % Кодировка входного документа;
                                    % при необходимости, вместо cp1251
                                    % можно указать cp866 (Alt-кодировка
                                    % DOS) или koi8-r.

\usepackage[english,russian]{babel} % Включение русификации, русских и
                                    % английских стилей и переносов
%%\usepackage{a4}
%%\usepackage{moreverb}
\usepackage{amsmath,amsfonts,amsthm,amssymb,amsbsy,amstext,amscd,amsxtra,multicol}
\usepackage{indentfirst}
\usepackage{verbatim}
\usepackage{tikz} %Рисование автоматов
\usetikzlibrary{automata,positioning}
%\usepackage{multicol} %Несколько колонок
\usepackage{graphicx}
\usepackage[colorlinks,urlcolor=blue]{hyperref}
\usepackage[stable]{footmisc}

%% \voffset-5mm
%% \def\baselinestretch{1.44}
\renewcommand{\theequation}{\arabic{equation}}
\def\hm#1{#1\nobreak\discretionary{}{\hbox{$#1$}}{}}
\newtheorem{Lemma}{Лемма}
\theoremstyle{definiton}
\newtheorem{Remark}{Замечание}
%%\newtheorem{Def}{Определение}
\newtheorem{Claim}{Утверждение}
\newtheorem{Cor}{Следствие}
\newtheorem{Theorem}{Теорема}
\theoremstyle{definition}
\newtheorem{Example}{Пример}
\newtheorem*{known}{Теорема}
\def\proofname{Доказательство}
\theoremstyle{definition}
\newtheorem{Def}{Определение}

%% \newenvironment{Example} % имя окружения
%% {\par\noindent{\bf Пример.}} % команды для \begin
%% {\hfill$\scriptstyle\qed$} % команды для \end






%\date{22 июня 2011 г.}
\let\leq\leqslant
\let\geq\geqslant
\def\MT{\mathrm{MT}}
%Обозначения ``ажуром''
\def\BB{\mathbb B}
\def\CC{\mathbb C}
\def\RR{\mathbb R}
\def\SS{\mathbb S}
\def\ZZ{\mathbb Z}
\def\NN{\mathbb N}
\def\FF{\mathbb F}
%греческие буквы
\let\epsilon\varepsilon
\let\es\varnothing
\let\eps\varepsilon
\let\al\alpha
\let\sg\sigma
\let\ga\gamma
\let\ph\varphi
\let\om\omega
\let\ld\lambda
\let\Ld\Lambda
\let\vk\varkappa
\let\Om\Omega
\def\abstractname{}

\def\R{{\cal R}}
\def\A{{\cal A}}
\def\B{{\cal B}}
\def\C{{\cal C}}
\def\D{{\cal D}}

%классы сложности
\def\REG{{\mathsf{REG}}}
\def\CFL{{\mathsf{CFL}}}


%%%%%%%%%%%%%%%%%%%%%%%%%%%%%%% Problems macros  %%%%%%%%%%%%%%%%%%%%%%%%%%%%%%%


%%%%%%%%%%%%%%%%%%%%%%%% Enumerations %%%%%%%%%%%%%%%%%%%%%%%%

\newcommand{\Rnum}[1]{\expandafter{\romannumeral #1\relax}}
\newcommand{\RNum}[1]{\uppercase\expandafter{\romannumeral #1\relax}}

%%%%%%%%%%%%%%%%%%%%% EOF Enumerations %%%%%%%%%%%%%%%%%%%%%

\usepackage{xparse}
\usepackage{ifthen}
\usepackage{bm} %%% bf in math mode
\usepackage{color}
%\usepackage[usenames,dvipsnames]{xcolor}

\definecolor{Gray555}{HTML}{555555}
\definecolor{Gray444}{HTML}{444444}
\definecolor{Gray333}{HTML}{333333}


\newcounter{problem}
\newcounter{uproblem}
\newcounter{subproblem}
\newcounter{prvar}

\def\beforPRskip{
	\bigskip
	%\vspace*{2ex}
}

\def\PRSUBskip{
	\medskip
}


\def\pr{\beforPRskip\noindent\stepcounter{problem}{\bf \theproblem .\;}\setcounter{subproblem}{0}}
\def\pru{\beforPRskip\noindent\stepcounter{problem}{\bf $\mathbf{\theproblem}^\circ$\!\!.\;}\setcounter{subproblem}{0}}
\def\prstar{\beforPRskip\noindent\stepcounter{problem}{\bf $\mathbf{\theproblem}^*$\negthickspace.}\setcounter{subproblem}{0}\;}
\def\prpfrom[#1]{\beforPRskip\noindent\stepcounter{problem}{\bf Задача \theproblem~(№#1 из задания).  }\setcounter{subproblem}{0} }
\def\prp{\beforPRskip\noindent\stepcounter{problem}{\bf Задача \theproblem .  }\setcounter{subproblem}{0} }

\def\prpvar{\beforPRskip\noindent\stepcounter{problem}\setcounter{prvar}{1}{\bf Задача \theproblem \;$\langle${\rm\Rnum{\theprvar}}$\rangle$.}\setcounter{subproblem}{0}\;}
\def\prpv{\beforPRskip\noindent\stepcounter{prvar}{\bf Задача \theproblem \,$\bm\langle$\bracketspace{{\rm\Rnum{\theprvar}}}$\bm\rangle$.  }\setcounter{subproblem}{0} }
\def\prv{\beforPRskip\noindent\stepcounter{prvar}{\bf \theproblem\,$\bm\langle$\bracketspace{{\rm\Rnum{\theprvar}}}$\bm\rangle$}.\setcounter{subproblem}{0} }

\def\prpstar{\beforPRskip\noindent\stepcounter{problem}{\bf Задача $\bf\theproblem^*$\negthickspace.  }\setcounter{subproblem}{0} }
\def\prdag{\beforPRskip\noindent\stepcounter{problem}{\bf Задача $\theproblem^{^\dagger}$\negthickspace\,.  }\setcounter{subproblem}{0} }
\def\upr{\beforPRskip\noindent\stepcounter{uproblem}{\bf Упражнение \theuproblem .  }\setcounter{subproblem}{0} }
%\def\prp{\vspace{5pt}\stepcounter{problem}{\bf Задача \theproblem .  } }
%\def\prs{\vspace{5pt}\stepcounter{problem}{\bf \theproblem .*   }
\def\prsub{\PRSUBskip\noindent\stepcounter{subproblem}{\sf \thesubproblem .} }
\def\prsubr{\PRSUBskip\noindent\stepcounter{subproblem}{\bf \asbuk{subproblem})}\;}
\def\prsubstar{\PRSUBskip\noindent\stepcounter{subproblem}{\rm $\thesubproblem^*$\negthickspace.  } }
\def\prsubrstar{\PRSUBskip\noindent\stepcounter{subproblem}{$\text{\bf \asbuk{subproblem}}^*\mathbf{)}$}\;}

\newcommand{\bracketspace}[1]{\phantom{(}\!\!{#1}\!\!\phantom{)}}

\DeclareDocumentCommand{\Prpvar}{ O{null} O{} }{
	\beforPRskip\noindent\stepcounter{problem}\setcounter{prvar}{1}{\bf Задача \theproblem
% 	\ifthenelse{\equal{#1}{null}}{  }{ {\sf $\bm\langle$\bracketspace{#1}$\bm\rangle$}}
%	~\!\!(\bracketspace{{\rm\Rnum{\theprvar}}}).  }\setcounter{subproblem}{0}
%	\;(\bracketspace{{\rm\Rnum{\theprvar}}})}\setcounter{subproblem}{0}
%
	\,{\sf $\bm\langle$\bracketspace{{\rm\Rnum{\theprvar}}}$\bm\rangle$}
	~\!\!\! \ifthenelse{\equal{#1}{null}}{\!}{{\sf(\bracketspace{#1})}}}.

}
%\DeclareDocumentCommand{\Prpvar}{ O{level} O{meta} m }{\prpvar}


\DeclareDocumentCommand{\Prp}{ O{null} O{null} }{\setcounter{subproblem}{0}
	\beforPRskip\noindent\stepcounter{problem}\setcounter{prvar}{0}{\bf Задача \theproblem
	~\!\!\! \ifthenelse{\equal{#1}{null}}{\!}{{\sf(\bracketspace{#1})}}
	 \ifthenelse{\equal{#2}{null}}{\!\!}{{\sf [\color{Gray444}\,\bracketspace{{\fontfamily{afd}\selectfont#2}}\,]}}}.}

\DeclareDocumentCommand{\Pr}{ O{null} O{null} }{\setcounter{subproblem}{0}
	\beforPRskip\noindent\stepcounter{problem}\setcounter{prvar}{0}{\bf\theproblem
	~\!\!\! \ifthenelse{\equal{#1}{null}}{\!\!}{{\sf(\bracketspace{#1})}}
	 \ifthenelse{\equal{#2}{null}}{\!\!}{{\sf [\color{Gray444}\,\bracketspace{{\fontfamily{afd}\selectfont#2}}\,]}}}.}

%\DeclareDocumentCommand{\Prp}{ O{level} O{meta} }

\DeclareDocumentCommand{\Prps}{ O{null} O{null} }{\setcounter{subproblem}{0}
	\beforPRskip\noindent\stepcounter{problem}\setcounter{prvar}{0}{\bf Задача $\bm\theproblem^* $
	~\!\!\! \ifthenelse{\equal{#1}{null}}{\!}{{\sf(\bracketspace{#1})}}
	 \ifthenelse{\equal{#2}{null}}{\!\!}{{\sf [\color{Gray444}\,\bracketspace{{\fontfamily{afd}\selectfont#2}}\,]}}}.
}

\DeclareDocumentCommand{\Prpd}{ O{null} O{null} }{\setcounter{subproblem}{0}
	\beforPRskip\noindent\stepcounter{problem}\setcounter{prvar}{0}{\bf Задача $\bm\theproblem^\dagger$
	~\!\!\! \ifthenelse{\equal{#1}{null}}{\!}{{\sf(\bracketspace{#1})}}
	 \ifthenelse{\equal{#2}{null}}{\!\!}{{\sf [\color{Gray444}\,\bracketspace{{\fontfamily{afd}\selectfont#2}}\,]}}}.
}


\def\prend{
	\bigskip
%	\bigskip
}




%%%%%%%%%%%%%%%%%%%%%%%%%%%%%%% EOF Problems macros  %%%%%%%%%%%%%%%%%%%%%%%%%%%%%%%



%\usepackage{erewhon}
%\usepackage{heuristica}
%\usepackage{gentium}

\usepackage[portrait, top=3cm, bottom=1.5cm, left=3cm, right=2cm]{geometry}

\usepackage{fancyhdr}
\pagestyle{fancy}
\renewcommand{\headrulewidth}{0pt}
\lhead{\fontfamily{fca}\selectfont {Основные алгоритмы, ФПМИ, 2022} }
%\lhead{ \bf  {ТРЯП. } Семинар 1 }
%\chead{\fontfamily{fca}\selectfont {Вариант 1}}
\rhead{\fontfamily{fca}\selectfont Домашнее задание 2, Павлов М.А.}
%\rhead{\small 01.09.2016}
\cfoot{}

\usepackage{titlesec}
\titleformat{\section}[block]{\Large\bfseries\filcenter {\setcounter{problem}{0}}  }{}{1em}{}


%%%%%%%%%%%%%%%%%%%%%%%%%%%%%%%%%%%%%%%%%%%%%%%%%%%% Обозначения и операции %%%%%%%%%%%%%%%%%%%%%%%%%%%%%%%%%%%%%%%%%%%%%%%%%%%% 
                                                                    
\newcommand{\divisible}{\mathop{\raisebox{-2pt}{\vdots}}}           
\let\Om\Omega


%%%%%%%%%%%%%%%%%%%%%%%%%%%%%%%%%%%%%%%% Shen Macroses %%%%%%%%%%%%%%%%%%%%%%%%%%%%%%%%%%%%%%%%
\newcommand{\w}[1]{{\hbox{\texttt{#1}}}}


\begin{document}



\pr Решите уравнения в целых числах, используя расширенный алгоритм Евклида. Внимание! Требуется найти все решения, а не только частное решение, которое находит алгоритм Евклида. Выведите самостоятельно формулу для общего решения или воспользуйтесь помощью литературы.

\prsubr $ 238 x + 385 y = 133 $

\begin{center}
	\begin{tabular}{ | c | c | c |}
		\hline
		a & b & $238x + 385y$ \\ \hline
		1 & 0 & 238 \\ \hline
		-1 & 1 & 147 \\ \hline
		2 & -1 & 91 \\ \hline
		-3 & 2 & 56 \\ \hline
		5 & -3 & 35 \\ \hline
		-8 & 5 & 21 \\ \hline
		13 & -8 & 14 \\ \hline
		-21 & 13 & 7 \\ \hline
		55 & -34 & 0 \\ \hline
	\end{tabular}
\end{center}

$\frac{133}{7} = 19 \Rightarrow x = -399, y = 247 \Rightarrow (-399, 247)$ -- \textbf{частное решение}.

\textbf{Общее решение:} $(55n - 399, -34n + 247)$

\prsubr $143x+121y=52$.

\begin{center}
	\begin{tabular}{ | c | c | c |}
		\hline
		a & b & $143x + 121y$ \\ \hline
		1 & 0 & 143 \\ \hline
		0 & 1 & 121 \\ \hline
		1 & -1 & 22 \\ \hline
		-5 & 6 & 11 \\ \hline
		11 & -13 & 0 \\ \hline
	\end{tabular}
\end{center}

НОД $(143, 121) = 11$, но 52 не делится на 11 без остатка, поэтому в целых числах уравнение \textbf{не имеет решений}.

\pr Решите сравнение $68x + 85 \equiv 0 \pmod{561} $ с помощью расширенного алгоритма Евклида. (Требуется найти все решения в вычетах)

\begin{displaymath}
	Solution

	68x \equiv -85 (mod 561)

	68x \equiv 476 (mod 561)

	68x + 561y = 1
\end{displaymath}

А такое уравнение мы умеем решать с помощью алгоритма Евклида!

\begin{center}
	\begin{tabular}{ | c | c | c |}
		\hline
		a & b & $68x + 561y$ \\ \hline
		0 & 1 & 561 \\ \hline
		1 & 0 & 68 \\ \hline
		-8 & 1 & 17 \\ \hline
		33 & -4 & 0 \\ \hline
	\end{tabular}
\end{center}

$-266x + 33y = 561$

$-266x \equiv 0 (mod 33)$

\textbf{Ответ:} $x = 33n, n \in {0, 1,... 16}$

\pr Вычислите $7^{13} \mod 167$, используя алгоритм быстрого возведения в степень.

$7^{13} \equiv 7 \cdot (7^6)^2 \equiv 7 \cdot ((7^3)^2)^2 \equiv 7 \cdot ((7 \cdot 7^2)^2)^2 \equiv 7 \cdot (81)^2 \equiv -2 (mod 167)$

\prend

\Pr[null][ДПВ 1.8] Доказать корректность рекурсивного алгоритма умножения Divide (раздел 1.1., рис. 1.2.) и получить верхнюю оценку на время работы.

Корректность следует из того, что $y \geq 1$, а также из того, что функция деления (без нуля в знаменателе) очень похожа на функцию умножения, корректность которой была доказана ранее.

Верхняя оценка для вычисления времени работы -- $O(n^2)$. Считается аналогично нахождения сложности работы алгоритма умножения, который был рассмотрен ранее.

\prend

\pr Функции $T_1(n)$ и $T_2(n)$ заданы рекуррентными формулами, известно что $T_i(1) = T_i(2) = T_i(3) = 1, i = 1,2$.

	\prsub Найдите асимтотику роста функции~$T_1(n) = T_1(n-1)+cn$ (при $n > 3$);

		$\Theta (n)$

	\prsub Докажите, что для функции $T_2(n) = T_2(n-1)+4T_2(n-3)$ (при $n > 3$) справедлива оценка $\log T_2(n) = \Theta(n)$.

		Это доказывается с помощью построения дерева рекурсивных вызовов: поскольку каждый вызов активирует вместе с собой ещё два вызова, то высота дерева будет равна двоичному логарифму от $n$.


	\prsubstar Найдите (точную) асимтотику роста функции $T_2(n)$.

\prend

\Pr[null][Шень \textbf{1.1.17}] Добавим в~алгоритм Евклида дополнительные
переменные~\w{u}, \w{v},~\w{z}:%
\begin{verbatim}
         m := a; n := b; u := b; v := a;
        {инвариант: НОД (a,b) = НОД (m,n); m,n >= 0 }
        while not ((m=0) or (n=0)) do begin
        | if m >= n then begin
        | | m := m - n; v := v + u;
        | end else begin
        | | n := n - m; u := u + v;
        | end;
        end;
        if m = 0 then begin
        | z:= v;
        end else begin {n=0}
        | z:= u;
        end;
\end{verbatim}
Докажите, что после исполнения алгоритма значение~\w{z}
равно удвоенному наименьшему общему кратному
чисел~\w{a},~\w{b}: $\w{z} = \w{2}\cdot\w{НОК(a,b)}$.

Для начала стоит отметить, что $m \cdot u + n \cdot v$ не меняется в ходе алгоритма (т.к. увеличиваются либо u, либо v, а уменьшаются либо m, либо n; и вначале равна ab.

Далее замечаем, что НОД(a, b) \cdot НОК(a, b) = $ab$.



\prend

\prstar Предложите $O(\sqrt{m}\log m)$ алгоритм нахождения длины периода десятичной дроби $\frac{n}{m}$. Докажите его корректность и оцените асимптотику.

\prend

\prstar Доказать, что \w{inv(i, p): return i > 1 ? -(p/i)*inv(p\%i, p) \% p : 1} возвращает обратный остаток, доказать, что работает за логарифм и развернуть рекурсию.

\prend

\prstar $f(1) = g(1) = 1$
$f(n) = a\cdot g(n-1) + b\cdot f(n-1)$
$g(n) = c\cdot g(n-1) + d\cdot f(n-1)$
где $a,b,c,d$ положительные константы. Предложите алгоритм вычисляющий $f(n)$ со сложностью $O(\log n)$ арифметических операций. 
\end{document}
  